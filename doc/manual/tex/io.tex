%%%documentation of the io submodule

The {\tt io} submodule provides classes for reading x-ray diffraction data in
various formats. 

\section{Reading SPEC files}

Working with spec files in xrutils can be done in two distinct ways. 
\begin{enumerate}
 \item parsing the spec file for scan headers; and parsing the data only when needed
 \item parsing the spec file for scan headers; parsing all data and dump them to an HDF5 file; reading the data from the HDF5 file. 
\end{enumerate}
Both methods have their pros and cons. For example when you parse the spec-files over a network connection you need to re-read the data again over the network if using method 1) whereas you can dump them to a local file with method 2). But you will parse data of the complete file while dumping it to the HDF5 file. 

Both methods work incremental, so they do not start at the beginning of the file when you reparse it, but start from the last position they were reading.

An working example for methode two is given in the following.

\begin{lstlisting}[caption=parsing a SPEC file and read data (1) or dumping the data to an HDF5 file (2)]
import tables
import xrutils as xu

# open spec file or use open SPECfile instance
try: s
except NameError:
    s = xu.io.SPECFile("sample_name.spec",path="./specdir")

# method (1)
scan10 = s[9] # Returns a SPECScan class
scan10.ReadData()
scan10data = scan10.data

# method (2)
h5file = "./h5dir/h5file.h5"
try: h5.isopen # open HDF5 file or use open one
except NameError:
    h5 = tables.openFile(h5file,mode='a')
else:
    if not h5.isopen: h5 = tables.openFile(h5file,mode='a')
s.Save2HDF5(h5) # save content of SPEC file to HDF5 file
# read data from HDF5 file
scan10data = h5.root.sample_name.scan_10.data.read()
\end{lstlisting}

\begin{lstlisting}[caption=reparse the SPEC file for new scans and reread the scans (1) or update the HDF5 file(2)]
...

s.Update() # reparse for new scans in open SPECFile instance

# reread data method (1)
scan10 = s[9] # Returns a SPECScan class
scan10.ReadData()
scan10data = scan10.data 

# reread data method (2)
s.Save2HDF5(h5) # save content of SPEC file to HDF5 file
# read data from HDF5 file
scan10data = h5.root.sample_name.scan_10.data.read()
\end{lstlisting}

\section{Reading EDF files}

EDF files are mostly used to store CCD frames at ESRF. This format is therefore used in combination with SPEC files. In an example the EDFFile class is used to parse the data from EDF files and store them to an HDF5 file. HDF5 if perfectly suited because it can handle large amount of data and compression.

\begin{lstlisting}[caption=script to parse and plot a reciprocal space map recorded with Seifert's XRD control software]
import tables 
import xrutils as xu
import numpy

specfile = "specdir/mo3211b.dat"
h5file = "h5dir/mo3211b.h5"
h5 = tables.openFile(h5file,mode='a')

s = xu.io.SPECFile(specfile,path=specdir)
s.Save2HDF5(h5) # save to hdf5 file

# read ccd frames from EDF files
for i in range(1,1000,1):
    efile = "edfdir/sample_%04d.edf" %i
    e = xu.io.edf.EDFFile(efile,path=specdir)
    e.ReadData()
    g5 = h5.createGroup(h5.root,"frelon_%04d" %i)
    e.Save2HDF5(h5,group=g5)

h5.close()
\end{lstlisting}

\section{Reading Bruker CCD data}

\section{Reading Radicon files}

Usage of Radicon files is deprecated. They were used by an old control software of Radicon. There are functions to read this format!

\section{Reading Seifert ASCII files}

Two functions to read ASCII data from Seifert's XRD control software are available. One which reads single scans most times recorded with the point detector and the other one for parsing files which contain multiple scans (e.g.: RSMs recorded with the PSD).

\begin{lstlisting}[caption=script to parse and plot a reciprocal space map recorded with Seifert's XRD control software]
import xrutils as xu
import matplotlib.pyplot as plt
import numpy

exp111 = xu.HXRD([1,1,-2],[1,1,1])
# 
data = xu.io.SeifertMultiScan('seifertmultiscan.nja','T','O')

om  = data.m2_pos[:,numpy.newaxis]*numpy.ones(data.int.shape)
tt  = data.sm_pos[numpy.newaxis,:]*numpy.ones(data.int.shape)

[qdummy,qy,qz] = exp111.Ang2Q(om,tt,0.)

gridder = xu.Gridder2D(300,300)
gridder(qy,qz,data.int)
INT = numpy.log10(gridder.gdata.transpose().clip(0.01,1e9))

plt.figure(); plt.clf()
plt.contourf(gridder.xaxis,gridder.yaxis,INT,50)
plt.xlabel(r'Qx'); plt.ylabel(r'Qz'); plt.colorbar()
\end{lstlisting}

\begin{lstlisting}[caption=script to parse and plot a single scan recorded with Seifert's XRD control software]
import xrutils as xu
import matplotlib.pyplot as plt

data = xu.io.SeifertScan('seifert.nja')

plt.figure(); plt.clf()
plt.plot(d.data[:,0],d.data[:,1],'k-')
plt.xlabel(r'scan axis (deg)'); plt.ylabel(r'Int (cps)'); plt.grid()
\end{lstlisting}